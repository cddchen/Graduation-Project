\chapter{绪论}
 
 \section{论文研究背景与意义}
\subsection{选题的背景}
对于许多实际生活中需要求得两个地方之间的路径和距离,并且考虑障碍物的影响,以提供后续决策等的帮助,例如房间当中两个窗户之间风的流向轨迹,在山体之间修路等问题的求解时需要计算出起点到终点之间的最短路径。对于小数据范围如房屋空间,不需要复杂的数学模型建模转换为二维模型以求解最短路径,可以在考虑一定误差范围内将障碍物转换为AABB包围盒,利用最短路径算法直接在三维空间中求解起点到终点的近似最短路径。
\subsection{选题的技术现状}
本算法通过将空间离散为若干个格子点,将障碍物近似为每一边都平行于一个坐标平面的简单六面体,通过常规最短路径算法:BFS广度优先搜索算法和A*寻路算法求解给定起点到终点不经过障碍物的的可行路径,再拟合成若干线段,通过检查线段是否穿过障碍物实现路径拟合,最后通过动态规划计算最优路径长度。
\subsection{选题的意义}
完成本选题,是为了设计并实现算法以解决实际项目中遇到的路径长度求解问题,提供近似最短路径和欧式距离以提供路径选择参考。
\section{国内外研究现状}
国内外对于三维最短路径算法的研究一般基于数字高程模型(Digital Elevation Model,简称DEM),是通过用一组有序数值阵列形式表示地面高程的数据集。对于DEM给出的地面高程信息,在测绘、水文、气象等应用地面模型的领域都取得了突破性进展。本文不利用DEM模型,而是将三维空间离散为若干个格点,再应用最短路径算法在格点间,求得起点格点到终点格点的最短路径后再将格点之间的路线拟合为线段的实现方式。
\section{论文组织结构}
本文针对三维空间路径规划问题开展研究,首先将障碍物的aabb包围盒以某个尺度离散为具体的格点,对格点进行不可访问标记处理;随后基于所给的起点,采用最短路径算法求得到终点所经过的格点序列,提出一种基SPFA和A*的最短路径算法实现,同时对比了基于不同最短路径算法之间的空间-时间开销。本文的组织结构如下:\\第一章:本章主要介绍研究课题的背景和意义,三维空间最短路径研究现状及发展的趋势,并讨论了国内外在相关方面的研究状况及其应用前景,给出本文的主要贡献、创新性和组织结构。\\第二章:本章主要介绍了最短路径模型、二维和三维空间最短路径的区别和联系、图的概念、存储方式并对比了不同存储方式的优缺点。最后,总结和阐述了三维空间最短路径的规划方法。\\第三章:本章主要对三维空间最短路径求解算法进行分类总结,提出将三维空间离散为格点后,在格点图上进行BFS、A*、SPFA等算法求解最短路径的效率和开销。\\第四章:对全文进行了总结,对未来基于三维空间的最短路径算法深入研究作出了展望。